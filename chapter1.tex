\section*{Chapter 1}
\subsection*{Preliminaries: Set theory and categories}
\setcounter{subsection}{1}

% Problem 1.1
\begin{problem}
  Locate a discussion of Russel's paradox, and understand it.
\end{problem}
\begin{solution}
  Russell's Paradox arises within naive set theory when considering the set
  $R$ of all sets that are not members of themselves. If $R \in R$, then by
  definition of $R$, $R \notin R$, a contradiction. If $R \notin R$, then $R$
  does not contain itself, hence $R$ must be a member of $R$,
  a contradiction.
\end{solution}

% Problem 1.2
\begin{problem}
  $\rhd$ Prove that if $\sim$ is an equivalence relation on a set $S$, then
  the corresponding family $\mathscr{P}_{\sim}$ defined in \S1.5 is indeed a
  partition of $S$; that is, its elements are nonempty, disjoint, and their
  union is $S$. [\S1.5]
\end{problem}
\begin{solution}

\end{solution}

% Problem 1.3
\begin{problem}
  $\rhd$ Given a partition $\mathscr{P}$ on a set $S$, show how to define a
  relation $\sim$ such that $\mathscr{P} = \mathscr{P}_{\sim}$. [\S1.5]
\end{problem}

% Problem 1.4
\begin{problem}
  How many different equivalence relations can be defined on the set
  $\{1,2,3\}?$
\end{problem}
